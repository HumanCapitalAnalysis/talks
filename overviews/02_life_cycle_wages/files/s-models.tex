%---------------------------------------------------------------------------------------------------
%---------------------------------------------------------------------------------------------------
\begin{frame}\begin{center}
	\LARGE\textbf{Models}
\end{center}\end{frame}
%---------------------------------------------------------------------------------------------------
%---------------------------------------------------------------------------------------------------
\begin{frame}\textbf{Implications}\vspace{0.3cm}

\begin{itemize}\setlength\itemsep{1em}
\item Investment
\item Investment in school and on the job
\item Search
\item $\hdots$
\end{itemize}

\end{frame}
%---------------------------------------------------------------------------------------------------
%---------------------------------------------------------------------------------------------------
\begin{frame}\begin{center}
	\LARGE\textit{Investment}
\end{center}\end{frame}
%---------------------------------------------------------------------------------------------------
%---------------------------------------------------------------------------------------------------
\begin{frame}\textbf{Implications}\vspace{0.3cm}

\begin{itemize}\setlength\itemsep{1em}
\item For a constant $R$, investment declines as the worker ages and approaches the end his working life.
\item Earnings rise along an optimal investment path. This is caused by two effects that reinforce each other; positive investment increases earning capacity and declining investment induces a rise in its utilization rate.
\end{itemize}

\end{frame}
%---------------------------------------------------------------------------------------------------
%---------------------------------------------------------------------------------------------------
\begin{frame}\textbf{Implications}\vspace{0.3cm}

\begin{itemize}\setlength\itemsep{1em}
\item If $R$ varies with time, workers that expect exogenous growth in their earning capacity invest at a higher rate and their wage rises at a higher pace. Investment declines if the rate of growth in the rental rate decreases.
\end{itemize}

\end{frame}
%---------------------------------------------------------------------------------------------------
%---------------------------------------------------------------------------------------------------
\begin{frame}\begin{center}
	\LARGE\textit{Investment in school \\ and on the job}
\end{center}\end{frame}
%---------------------------------------------------------------------------------------------------
%---------------------------------------------------------------------------------------------------
\begin{frame}\textbf{Implications}\vspace{0.3cm}

\begin{itemize}\setlength\itemsep{1em}
\item Specialization in schooling occurs, if at all, in the first phase of life. It is followed by a period of investment on the job. In the last phase of the life cycle, there is no investment at all.
\item During the schooling period, there are no earnings, yet human capital is accumulated at the maximal rate $(1 + \gamma )$. During the period of investment on the job, earnings are positive and growing. In the last phase (if it exists), earnings are constant.
\end{itemize}

\end{frame}
%---------------------------------------------------------------------------------------------------
%---------------------------------------------------------------------------------------------------
\begin{frame}\textbf{Implications}\vspace{0.3cm}

\begin{itemize}\setlength\itemsep{1em}
\item A worker leaves school at the first period in which (10) is reversed. At this point it must be the case that $l_t^* < 1$, which means that at the time of leaving school, earnings must jump to a positive level. This realistic feature is present only because we assume different production (and cost) functions in school and on the job, whereby accumulation in school is faster but requires a larger sacrifice of current earnings.
\end{itemize}

\end{frame}
%---------------------------------------------------------------------------------------------------
%---------------------------------------------------------------------------------------------------
\begin{frame}\textbf{Implications}\vspace{0.3cm}

\begin{itemize}\setlength\itemsep{1em}
\item A person with a larger initial stock of human capital, $K > 0$ , will stay in school for a shorter period and spend more time investing on the job. He will have higher earnings and the same earnings growth throughout life.
\item A person with a larger scholastic learning ability, $\gamma$, will stay in school for a longer period and spend less time investing on the job. He will also have higher earnings and the same earning growth throughout life.
\end{itemize}

\end{frame}
%---------------------------------------------------------------------------------------------------
%---------------------------------------------------------------------------------------------------
\begin{frame}\begin{center}
	\LARGE\textit{Search}
\end{center}\end{frame}
%---------------------------------------------------------------------------------------------------
%---------------------------------------------------------------------------------------------------
\begin{frame}\textbf{Implications}\vspace{0.3cm}

\begin{itemize}\setlength\itemsep{1em}
\item A job has an option value to the worker. In particular, he can maintain his current wage and move away when he gets a better offer. Consequently, earnings rise whenever the worker switches jobs and remain constant otherwise.
\item The higher the worker's current wage, the more valuable is the current job; hence, the offers that the workers accepts must exceed a higher reservation value. Therefore, the quit rate and the expected wage growth decline as the worker accumulates work experience and climbs up the occupational ladder.
\end{itemize}

\end{frame}
%---------------------------------------------------------------------------------------------------
%---------------------------------------------------------------------------------------------------
\begin{frame}\textbf{Implications}\vspace{0.3cm}

\begin{itemize}\setlength\itemsep{1em}
\item A straight-forward extension is to add involuntary separations. Such separations are usually associated with wage reduction and are more likely to occur at the end of the worker's career, which may explain the reduction in average wages towards the end of the life cycle.
\end{itemize}

\end{frame}
