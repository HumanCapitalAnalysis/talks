\begin{frame}\begin{center}
\LARGE\textbf{Housekeeping}
\end{center}\end{frame}

\begin{frame}[c]\frametitle{Setup for Python tutorial}
\begin{enumerate}
    \item download the files required for the tutorial from \href{https://www.dropbox.com/sh/0onu61t2olg2gwz/AAAtqiANuKODdBDdPlGTYtXja?dl=0}{here}, unzip and place them into a folder in your user directory.
    \item download the installer for Python 3.6 from \url{https://www.anaconda.com/download/} and run it
    \begin{itemize}
        \item \textit{If you are asked whether Anaconda and its paths should be added to your system's PATH or not, choose the option to add them}
    \end{itemize}
    \item start the Jupyter notebook in one of two ways
    \begin{enumerate}
        \item use terminal, shell, cmd, powershell to navigate to your project's folder and enter \texttt{jupyter notebook}
        \item start Jupyter via the Anaconda Navigator
    \end{enumerate}
    \item make sure that you can navigate to the tutorial folder inside the opened tab in your browser
    \item (optional) Start a new notebook by clicking on \texttt{New} in the top right corner and select Python 3
\end{enumerate}
\end{frame}

\begin{frame}[c]\frametitle{Suggestions for Python tutorial}
If you have suggestions for topics, problems and you want to see how they are handled in Python, email me at \href{mailto:tobiasraabe@uni-bonn.de}{tobiasraabe@uni-bonn.de}. As I will work with the NLSY79, problems regarding this dataset can be directly addressed. Otherwise, I will likely find a similar problem in the NLSY79 and show a solution.
\end{frame}

\begin{frame}[c]\frametitle{Setup for the introduction to NLSY}
Visit \url{https://www.nlsinfo.org/investigator/pages/login.jsp} and create an account. Registration does not require your real name, only an email address and a username.
\end{frame}


\begin{frame}\textbf{Some Questions}\vspace{0.3cm}
\begin{itemize}\setlength\itemsep{1em}
\item How are things going with your projects?
\end{itemize}
\end{frame}
