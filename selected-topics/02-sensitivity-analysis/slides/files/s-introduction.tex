% !TEX root = ../main.tex
%-------------------------------------------------------------------------------
%-------------------------------------------------------------------------------
\begin{frame} I draw on the material presented in:

\begin{itemize}\setlength\itemsep{1em}
  \item \bibentry{Saltelli.2004}
  \item \bibentry{Saltelli.2008}
\end{itemize}

\end{frame}
%-------------------------------------------------------------------------------
%-------------------------------------------------------------------------------
\begin{frame}\textbf{Definitions}\vspace{0.3cm}

\textit{Sensitivity analysis} The study of how uncertainty in the output of a model (numerical or otherwise) can be apportioned to different sources of uncertainty in the model input

\begin{itemize}
  \item uncertainty propagation
\end{itemize}
\end{frame}
%-------------------------------------------------------------------------------
%-------------------------------------------------------------------------------
\begin{frame}\textbf{Selected issues}\vspace{0.3cm}

\begin{itemize}\setlength\itemsep{1em}
  \item computational challenges
  \item deterministic vs. probabilistic
  \item independent vs. dependent
  \item global vs. local
\end{itemize}

\end{frame}
%-------------------------------------------------------------------------------
%-------------------------------------------------------------------------------
\begin{frame}\textbf{Selected issues}\vspace{0.3cm}

\begin{itemize}\setlength\itemsep{1em}
  \item quantitative vs. qualitative
  \item interaction vs. dependence
  \item full model vs. surrogate
\end{itemize}

\end{frame}
%-------------------------------------------------------------------------------
%-------------------------------------------------------------------------------
\begin{frame}\textbf{Settings}\vspace{0.3cm}

\begin{itemize}\setlength\itemsep{1em}
  \item \textit{factor prioritization}, i.e. which factor is the one that, if determined (i.e., fixed to its true, albeit unknown, value), would lead
to the greatest reduction in the variance of the output.
  \item \textit{factor fixing}, i.e. which factor or the subset of input factors that we can fix at any given value over their range of uncertainty without significantly reducing the output variance.
\end{itemize}

\end{frame}
%-------------------------------------------------------------------------------
%-------------------------------------------------------------------------------
\begin{frame}\textbf{Sensitivity analysis in economics}\vspace{0.3cm}

  \begin{itemize}
  \item \bibentry{Harenberg.2019}
  \end{itemize}

\end{frame}
