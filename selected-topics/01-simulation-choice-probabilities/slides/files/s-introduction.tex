%-------------------------------------------------------------------------------
%-------------------------------------------------------------------------------
\begin{frame} I heavily draw on the material presented in:

\begin{itemize}
\item \bibentry{Train.2009}
\end{itemize}

\end{frame}
%-------------------------------------------------------------------------------
%-------------------------------------------------------------------------------
\begin{frame}\textbf{Probit setup}\vspace{0.3cm}

\begin{align*}
U_{nj} = V_{nj} + \epsilon_{nj} \forall j \\
\epsilon^\prime_{n} = (\epsilon_{n1}, \hdots, \epsilon_{nJ})
\end{align*}
The (most) approaches and issues we discus have general applicability.

\end{frame}
%-------------------------------------------------------------------------------
%-------------------------------------------------------------------------------
\begin{frame}\textbf{Choice probability}\vspace{0.3cm}

\begin{align*}
  P_{ni} & = P(V_{ni} + \epsilon_{ni} > V_{nj} + \epsilon_{nj} \forall j \neq i) \\
         & = \int \Ind (V_{ni} + \epsilon_{ni} > V_{nj} + \epsilon_{nj} \forall j \neq i) \phi(\epsilon_n)d\epsilon_n
\end{align*}

\end{frame}
%-------------------------------------------------------------------------------
%-------------------------------------------------------------------------------
\begin{frame}\vspace{0.3cm}

The choice probabilities do not have a closed form expression and must be approximated numerically

\begin{itemize}\setlength\itemsep{1em}
  \item quadrature
  \item Monte Carlo methods
  \begin{itemize}\setlength\itemsep{1em}
    \item crude accept-reject simulator
    \item smoothed accept-reject simulator
  \end{itemize}
\end{itemize}
\end{frame}
%-------------------------------------------------------------------------------
%-------------------------------------------------------------------------------
\begin{frame}\textbf{Crude accept-reject simulator}\vspace{0.3cm}
\begin{enumerate}\setlength\itemsep{1em}
  \item Draw $J$ values from the multivariate normal distribution to sample $\epsilon^r_{n}$.
  \item Calculate the simulated utilities $U^r_{nj}$ for all alternatives.
  \item Determine alternative with the highest utility and calculate $I^r = 1$ if $U^r_{nj}$ is maximum.
\end{enumerate}
\end{frame}
%-------------------------------------------------------------------------------
%-------------------------------------------------------------------------------
\begin{frame}\textbf{Crude accept-reject simulator}\vspace{0.3cm}
\begin{enumerate}\setlength\itemsep{1em}
  \item Repeat the steps above $R$ times
\end{enumerate}

The simulated probability is the number of accepts divided by the number of repetitions: $\hat{P}_{ni} = \frac{1}{R} \sum^R_{r=1} I^r$.

\end{frame}
%-------------------------------------------------------------------------------
%-------------------------------------------------------------------------------
\begin{frame}\textbf{Issues}\vspace{0.3cm}
\begin{itemize}\setlength\itemsep{1em}
  \item simulation of zero probability
  \item simulated probability is step function\\
\end{itemize}

We will explore issues in estimation using a Python notebook: \url{http://bit.ly/2WGjWNI}.

\end{frame}
%-------------------------------------------------------------------------------
%-------------------------------------------------------------------------------
\begin{frame}\textbf{Smooth accept-reject simulator}\vspace{0.3cm}

The smoothed AR simulator mitigates these difficulties with is to replace the 1 - 0 AR indicator with a smooth, strictly positive function.

We simply add an addition step (4):

\begin{align*}
  S^r = \frac{\exp^{\tfrac{U^r_{ni}}{\lambda}}}{\sum_J\exp^{\tfrac{U^r_{nj}}{\lambda}}}
\end{align*}

The simulated probability is the number of accepts divided by the number of repetitions: $\hat{P}_{ni} = \frac{1}{R} \sum^R_{r=1} S^r$.

\end{frame}


\begin{frame}

We will explore issues in estimation using a Python notebook: \url{http://bit.ly/2WGjWNI}.

\end{frame}
